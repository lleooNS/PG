% ==============================================================================
% Anteprojeto de PG - Nome do Aluno
% Capítulo 3 - Método
% ==============================================================================

\section{Método de Trabalho}
\label{sec-metodo}

%---Apagar---
%Para atingir o objetivo geral apresentado na seção anterior, os seguintes passos são realizados:

%\begin{enumerate}
	
%	\item Descreva aqui as tarefas feitas em PG-1 e a serem feitas em PG-2.
	
%\end{enumerate}

De acordo com a seção anterior, para que os objetivos apresentados sejam alcançados, será necessário realizar os seguintes passos:

\begin{enumerate}

    \item Revisão bibliográfica e pesquisa: leitura dos padrões de Projeto de Sistemas de Software \cite{falbo:pss18}, análise dos temas de Engenharia de Software \cite{falbo:es14}, Engenharia de Requisitos~\cite{falbo:er17} e entendimento do método FrameWeb~\cite{souza:masterthesis07} para realizar a aplicação dos \textit{frameworks} GWT e Grails;
    \item Estudo do sistema SCAP: organização das informações referentes aos requisitos levantados por \citeonline{duarte-pg14} e \citeonline{prado-pg15};
    \item Definição da documentação do projeto: através do uso do método FrameWeb, elaboração da arquitetura do projeto para cada \textit{framework} utilizado;
    \item Desenvolvimento da implementação: geração de duas novas implementações do SCAP através do uso dos \textit{frameworks} descritos no projeto; 
    \item Redação da monografia: utilização do template abnTeX\footnote{\url{https://www.abntex.net.br/}} para a escrita da monografia em \latex\footnote{\url{https://www.latex-project.org/}} seguindo os requisitos das normas da ABNT (Associação Brasileira de Normas Técnicas);
    \item Apresentação do Projeto: apresentação final da monografia e demonstração do sistema SCAP com os \textit{frameworks} propostos.

\end{enumerate}    
    