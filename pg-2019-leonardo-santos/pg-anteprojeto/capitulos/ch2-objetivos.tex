% ==============================================================================
% Anteprojeto de PG - Nome do Aluno
% Capítulo 2 - Objetivos do Trabalho
% ==============================================================================

\section{Objetivos do Trabalho}
\label{sec-objetivos}

%---Apagar---
%Descreva o objetivo geral do trabalho e em seguida apresente uma lista de subobjetivos. Objetivos devem ser escritos como algo a ser alcançado e não como tarefas (algo a ser feito).
%------------

A execução deste trabalho tem como objetivo geral a aplicação do método FrameWeb (SOUZA, 2007), realizando uma inovação da implementação do SCAP (Sistema de Controle de Afastamentos de Professores), utilizando os \textit{frameworks} GWT e Grails, uma vez que os requisitos foram levantados por Duarte (2014) e reformulados por Prado (2015), verificando como o método se comporta e apresentando a devida evolução. A partir dos conhecimentos adquiridos no curso de Ciência da Computação será possível aplicar os conceitos aprendidos para que o objetivo seja alcançado.

O objetivo geral pode ser subdividido nos seguintes objetivos específicos:

\begin{itemize}

	\item Utilizar os \textit{frameworks} GWT e Grails para realizar uma nova implementação do SCAP, aproveitando os seus requisitos. Nesse objetivo será possível aplicar os conceitos de Engenharia de Requisitos e Engenharia de Software;
	\item Aplicar o método FrameWeb para definir a documentação da arquitetura do projeto de sistema para cada \textit{framework};
    \item Realizar o estudo dos requisitos já levantados e efetuar uma comparação entre os frameworks que já foram utilizados em versões anteriores.

\end{itemize}
