% ==============================================================================
% Anteprojeto de PG - Nome do Aluno
% Capítulo 2 - Objetivos do Trabalho
% ==============================================================================

\section{Objetivos do Trabalho}
\label{sec-objetivos}

%---Apagar---
%Descreva o objetivo geral do trabalho e em seguida apresente uma lista de subobjetivos. Objetivos devem ser escritos como algo a ser alcançado e não como tarefas (algo a ser feito).
%------------

A execução deste trabalho tem como objetivo geral a aplicação do método FrameWeb \cite{souza:masterthesis07}, realizando uma inovação da implementação do SCAP, utilizando os \textit{frameworks} GWT e Grails. Uma vez que os requisitos foram levantados por \citeonline{duarte-pg14} e reformulados por \citeonline{prado-pg15}, será possível verificar como o método se comporta, apresentando a devida evolução.

O objetivo geral pode ser subdividido nos seguintes objetivos específicos:

\begin{itemize}

	\item Utilizar os \textit{frameworks} GWT e Grails para realizar uma nova implementação do SCAP, aproveitando os seus requisitos. Nesse objetivo será possível aplicar os conceitos de Engenharia de Requisitos, Projeto de Sistemas de Software e Engenharia de Software;
	\item Aplicar o método FrameWeb para definir a documentação da arquitetura do projeto de sistema para cada \textit{framework};
    \item Realizar o estudo dos requisitos já levantados e efetuar uma comparação entre os \textit{frameworks} que já foram utilizados em versões anteriores.

\end{itemize}
