% ==============================================================================
% Anteprojeto de PG - Nome do Aluno
% Capítulo 1 - Introdução
% ==============================================================================

\section{Introdução}
\label{sec-intro}

%Descreva o contexto e a motivação para o projeto. Exemplos de referência: \citeonline{guarino-et-al:hobook09} (in-line) ou~\cite{guarino-et-al:hobook09} \cite{souza:masterthesis07} \citeonline{souza:masterthesis07}.

Antigamente, os servidores não comportavam páginas \textit{Web} que eram desenvolvidas de forma mais robusta. Funcionando apenas com páginas estáticas, realizar manutenções e o controle das funcionalidades era uma tarefa muito difícil. Com o desenvolvimento de novas tecnologias a partir da criação da \textit{World Wide Web} (WWW) e através do surgimento de novas linguagens de programação para a \textit{Web}, os servidores tiveram que ser adaptados para se tornarem mais robustos. Com a infraestrutura de software modificada, os servidores começaram a admitir páginas dinâmicas que passaram a ser carregadas de conteúdos que antes não eram suportados e de diversos efeitos.

Ainda nessa constante evolução, sistemas projetados para serem utilizados a partir de um navegador ou de aplicativos, puderam ser criados de uma maneira em que os servidores processassem e retornassem as informações para os visitantes através da internet, surgindo assim o conceito de \textit{WebApp}, ou aplicação \textit{Web}. Lojas virtuais e sistemas de fornecimento entre empresas são exemplos de \textit{websites} que puderam ser desenvolvidos por meio de aplicações \textit{Web}. Desta maneira, o foco deste trabalho serão os Sistemas de Informação Baseados na \textit{Web (Web-based Information Systems – WISs)}, que são enquadrados em uma categoria específica de \textit{WebApps}. Esses sistemas são caracterizados como sistemas de informação tradicionais, mas então disponíveis na Internet.

Quando falamos sobre o desenvolvimento de \textit{WebApps} atuais, em particular os WISs, temos que entender que utilizar a Engenharia de Software é uma tarefa fundamental. Os aspectos relacionados ao estabelecimento de técnicas, processos, métodos, ferramentas e ambientes de suporte ao desenvolvimento de software são tratados de forma clara pela Engenharia de Software \cite{falbo:es14}. Trazendo alguns benefícios como por exemplo a compatibilidade entre plataformas e a facilidade de gerenciamento, as aplicações \textit{Web} por meio dos servidores, se tornaram indispensáveis para manterem os serviços e as aplicações disponíveis em qualquer parte do mundo através da Internet.

No meio de tantas criações e adaptações tecnológicas, surge o uso de \textit{frameworks} para aplicações \textit{Web}. Se tornando uma das ferramentas mais importantes para o desenvolvimento de \textit{WebApps}, os \textit{frameworks} passaram a auxiliar na encapsulação das funcionalidades de alto nível com maior eficiência e agilidade, fazendo com que a maior parte do tempo e do trabalho fossem economizados. Visando propor uma abordagem diferenciada para a construção de sistemas para \textit{Web}, surge então o método FrameWeb (\textit{Framework-based Design Method for Web Engineering}) \cite{souza:masterthesis07}. Com o intuito de utilizar diversos \textit{frameworks}, o método FrameWeb busca agilizar o desenvolvimento de aplicações \textit{Web}, minimizando as tarefas realizadas nas fases que a Engenharia de Software determina.

Durante todo o processo, a linguagem de modelagem UML (Unifield Modeling Language) \cite{booch-et-al:u06} se torna presente nas tarefas realizadas. Sugerindo um processo de software orientado a objetos, o método FrameWeb engloba algumas atividades que são bastantes comuns nos processos de software, como levantamento de requisitos, análise, projeto, codificação, testes e implantação \cite{souza:masterthesis07}. O método não impõe processos específicos de desenvolvimento e deixa os desenvolvedores livres para realizarem adaptações.

Utilizando uma maneira para exemplificar e demonstrar a aplicação do método FrameWeb, \citeonline{duarte-pg14} desenvolveu uma aplicação Web --- um Sistema de Controle de Afastamento de Professores (SCAP) --- em seu trabalho de conclusão de curso. Esse sistema foi criado para apoiar um departamento de universidade a realizar um controle das solicitações de afastamento de seus professores efetivos. Utilizando os requisitos que foram levantados por \citeonline{duarte-pg14} e posteriormente analisados por \citeonline{prado-pg15}, este trabalho tem com tarefa fundamental a implementação do SCAP aplicando dois \textit{frameworks Web} diferentes para que seja possível realizar a verificação da eficiência do método FrameWeb.