% ==============================================================================
% Anteprojeto de PG - Nome do Aluno
% Capítulo 1 - Introdução
% ==============================================================================

\section{Introdução}
\label{sec-intro}

%Descreva o contexto e a motivação para o projeto. Exemplos de referência: \citeonline{guarino-et-al:hobook09} (in-line) ou~\cite{guarino-et-al:hobook09} \cite{souza:masterthesis07} \citeonline{souza:masterthesis07}.

Antigamente, os servidores não comportavam páginas web que eram desenvolvidas de forma mais robusta. Funcionando apenas com páginas estáticas, realizar manutenções e o controle das funcionalidades era uma tarefa muito difícil. Com o desenvolvimento de novas tecnologias a partir da criação da \textit{World Wide Web} (WWW) e através do surgimento de novas linguagens de programação para a Web, os servidores tiveram que ser adaptados para se tornarem mais robustos. Com a infraestrutura de software modificada, os servidores começaram a admitir páginas dinâmicas que passaram a ser carregadas de conteúdos que antes não eram suportados e de diversos efeitos.

Ainda nessa constante evolução, sistemas projetados para serem utilizados a partir de um navegador ou de aplicativos, puderam ser criados de uma maneira em que os servidores processassem e retornassem as informações para os visitantes através da internet, surgindo assim o conceito de \textit{WebApp}, ou aplicação \textit{Web}. Lojas virtuais e sistemas de fornecimento entre empresas são exemplos de \textit{websites} que puderam ser desenvolvidos por meio de aplicações \textit{Web}. Desta maneira, o foco deste trabalho serão os Sistemas de Informação Baseados na \textit{Web (Web-based Information Systems – WISs)}, que são enquadrados em uma categoria específica de \textit{WebApps}. Esses sistemas são caracterizados como sistemas de informação tradicionais, mas então disponíveis na Internet.

Quando falamos sobre o desenvolvimento de WebApps atuais, em particular os WISs, temos que entender que utilizar a Engenharia de Software é uma tarefa fundamental. Os aspectos relacionados ao estabelecimento de técnicas, processos, métodos, ferramentas e ambientes de suporte ao desenvolvimento de software são tratados pela Engenharia de Software \cite{falbo:es14}. Trazendo alguns benefícios como por exemplo a compatibilidade entre plataformas e a facilidade de gerenciamento, as aplicações Web, por meio dos servidores, se tornaram indispensáveis para manterem os serviços e as aplicações disponíveis em qualquer parte do mundo através da Internet. 