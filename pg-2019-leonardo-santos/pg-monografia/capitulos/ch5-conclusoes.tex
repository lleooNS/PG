% ==============================================================================
% TCC - Nome do Aluno
% Capítulo 5 - Considerações Finais
% ==============================================================================
\chapter{Considerações Finais}
\label{sec-conclusoes}

O capítulo apresenta as conclusões que foram obtidas através da aplicação do método FrameWeb, o uso da ferramenta FrameWeb \textit{Editor} para a geração dos modelos, a utilização do \textit{framework} Grails, o aprendizado da linguagem Apache Groovy, os problemas enfrentados e por fim, uma sugestão para trabalhos futuros.  

Para que os objetivos fossem alcançados, foi preciso combinar todo o conhecimento adquirido durante o curso, reunindo o aprendizado obtido nas disciplinas de Banco de Dados, Linguagens de Programação, Engenharia de \textit{Software}, Desenvolvimento \textit{Web} e \textit{Web} Semântica, entre outras. Através deste conhecimento, foi possível estabelecer a melhor estratégia para que os problemas enfrentados durante o projeto fossem sanados, gerando soluções que se tornaram bastante satisfatórias.  

A versão do sistema SCAP descrita no decorrer deste projeto foi implementada utilizando a linguagem Apache Groovy que ainda não tinha sido utilizada em versões anteriores. Por este motivo, é importante ressaltar que não foi utilizado nenhum código que estava presente nas versões passadas. Assim, foi necessário realizar o estudo e aprendizagem da linguagem Groovy e do \textit{framework} Grails para que eles pudessem ser utilizados. O aprendizado de uma nova linguagem é um desafio que proporciona um retorno muito grande, agregando muita experiência que pode ser aplicada em diversas áreas.

Com a utilização de um \textit{framework}, a implementação do sistema SCAP se tornou uma tarefa bem mais simples do que se fosse necessário implementar todo o código sem ter uma base de início. Além de fornecer a geração de boa parte do código necessário para o funcionamento do sistema, o uso de um \textit{framework} ainda proporciona a disponibilização de diversas bibliotecas que podem ser incluídas no projeto a qualquer momento, diminuindo o tempo e as complexidades do desenvolvimento.  

Durante o aprendizado para a utilização do \textit{framework} Grails, algumas dificuldades foram enfrentadas. Apesar do \textit{framework} disponibilizar uma rica documentação, alguns fatores são explicados de forma superficial, como por exemplo, a conexão com um banco de dados relacional. Para essa tarefa foi necessário recorrer aos fóruns da comunidade do \textit{framework}, assim como vídeos tutoriais encontrados na \textit{internet}. Outra dificuldade enfrentada foi a incompatibilidade entre as versões do \textit{framework}. Basta atualizar a versão e o projeto deixa de funcionar, sendo necessário realizar muitas alterações para que os erros sejam solucionados.

Depois que~\citeonline{souza:masterthesis07} estabeleceu uma organização para os \textit{frameworks} em categorias diferentes, realizar a escolha de um \textit{framework} que se enquadre na execução das propostas de um projeto se tornou uma tarefa bem mais fácil. Com a aplicação do método FrameWeb na fase de projeto arquitetural, a criação de modelos de projeto se aproxima muito da implementação do sistema, definindo uma arquitetura básica e deixando os desenvolvedores livres para adotarem as técnicas mais adequadas para as outras etapas do processo.

Para a criação dos modelos que são propostos pelo método FrameWeb, foi utilizada a ferramenta FrameWeb \textit{Editor}~\cite{campos-souza:webmedia17}. Nesta etapa, foi necessário realizar a instalação e configuração da ferramenta, seguindo o tutorial~\cite{souza:ftt21} referente a ela. As configurações são um pouco extensas e devem ser realizadas com bastante atenção para evitar possíveis erros. No tutorial também são listadas algumas dicas para resolver erros que podem aparecer durante as configurações.

A utilização do FrameWeb \textit{Editor} não é muito intuitivo no início, mas com relação as funcionalidades, a ferramenta cumpre o seu propósito. É possível realizar a criação dos quatro tipos básicos de modelos FrameWeb, fazendo verificações nos modelos e impedindo ações inválidas. Como toda ferramenta criada recentemente, ainda é necessário realizar correções em alguns erros, como por exemplo, o salvamento das associações entre os elementos contidos no modelo de navegação. Com o lançamento de novas versões da ferramenta, certamente esses erros serão solucionados.

Realizando a aplicação do método FrameWeb, claramente a sua eficiência e eficácia ficaram comprovadas. Por meio dos modelos que foram gerados, a implementação do código do sistema através do \textit{framework} Grails foi agilizada, sendo necessário apenas seguir com as etapas do projeto. Com a divisão do sistema em camadas, a organização e a modularização tornaram os componentes do SCAP bem mais fáceis de serem aperfeiçoados ou substituídos.

Em termos gerais, pertencente a categoria de Controlador Frontal, o \textit{framework} Grails foi considerado uma boa escolha para o desenvolvimento de aplicações \textit{Web}. Assim, como contribuição para o FrameWeb, ele pode ser incluído na lista de \textit{frameworks} que o método oferece suporte.                          

\section{Trabalhos Futuros}
\label{sec-conclusoes-trabalhos-futuros}

Existem muitos cenários onde um aplicativo \textit{mobile} traz melhor experiência para o usuário e satisfaz melhor suas expectativas, sobretudo com relação à usabilidade. Uma boa estratégia é começar sempre pela versão \textit{Web}, garantindo o acesso universal e multiplataforma. Conforme as necessidades forem surgindo, é possível investir em aplicativos específicos de plataformas com recursos e experiências nativas~\cite{lopes:awm14}.

De acordo com~\citeonline{lopes:awm14}, um aplicativo \textit{mobile} tem acesso direto ao hardware do aparelho e a recursos do sistema operacional. Consegue se integrar com funções avançadas e a outros aplicativos. Pode manipular o funcionamento do aparelho e até substituir ou complementar funções nativas. Já uma \textit{WebApp} roda enjaulada dentro do navegador e, por razões de segurança, não tem acesso direto à plataforma nativa.

Por estes motivos, como sugestão para trabalhos futuros, seria interessante realizar o desenvolvimento de um aplicativo \textit{mobile} que representasse o sistema SCAP. Após a aplicação do método FrameWeb para a construção de novos modelos, o próximo \textit{framework} utilizado ficaria responsável por controlar o \textit{back-end} da aplicação. Reunindo todas as funcionalidades, a usabilidade certamente seria ampliada, gerando pontos positivos para os seus usuários.   
