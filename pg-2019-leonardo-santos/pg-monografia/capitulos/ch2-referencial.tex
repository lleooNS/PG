% ==============================================================================
% TCC - Nome do Aluno
% Capítulo 2 - Referencial Teórico
% ==============================================================================
\chapter{Referencial Teórico}
\label{sec-referencial}

Descrever um pouco sobre o que vai ter no capítulo ...

\section{Engenharia \textit{Web}}
\label{sec-ref-engenharia-web}


A \textit{Web} é uma ferramenta que dispensa apresentações, pois ela já está familiarizada entre a maioria das pessoas e se encontra presente no dia a dia e em quase todas as áreas, podendo ser acessada através de muitos hardwares diferentes. Inicialmente, o conteúdo de \textit{websites} era apenas textual, estático e não existia a presença de animações, sons, imagens ou conteúdo gerado de maneira dinâmica para cada tipo de usuário. A preocupação dos desenvolvedores permanecia em torno da simplicidade de apenas visualizar as informações sem complexidades.

Com a evolução de forma acelerada da \textit{web}, surgiu a necessidade de mudanças significativas na maneira como os \textit{websites} eram criados. Os \textit{websites} passaram a englobar diversos conteúdos e funções complexas, adicionando centenas ou milhares de objetos em seu contexto. Sendo assim, quando toda a adição desses conteúdos começou a gerar um impacto direto no sucesso dos negócios, os projetos \textit{web} deixaram de ser tratados de maneira superficial~\cite{pressman:es11}. 	







       