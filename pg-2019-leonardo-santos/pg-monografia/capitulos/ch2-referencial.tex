% ==============================================================================
% TCC - Nome do Aluno
% Capítulo 2 - Referencial Teórico
% ==============================================================================
\chapter{Referencial Teórico}
\label{sec-referencial}

Descrever um pouco sobre o que vai ter no capítulo ...

\section{Engenharia \textit{Web}}
\label{sec-ref-engenharia-web}


A \textit{Web} é uma ferramenta que dispensa apresentações, pois ela já está familiarizada entre a maioria das pessoas, se encontra presente no dia a dia e em quase todas as áreas, podendo ser acessada através de muitos hardwares diferentes. Inicialmente, o conteúdo de \textit{websites} era apenas textual, estático e não existia a presença de animações, sons, imagens ou conteúdo gerado de maneira dinâmica para cada tipo de usuário. A preocupação dos desenvolvedores permanecia em torno da simplicidade de apenas visualizar as informações sem complexidades.

Com a evolução de forma acelerada da \textit{web}, surgiu a necessidade de mudanças significativas na maneira como os \textit{websites} eram criados. Os \textit{websites} passaram a englobar diversos conteúdos e funções complexas, adicionando centenas ou milhares de objetos em seu contexto. Sendo assim, quando toda a adição desses conteúdos começou a gerar um impacto direto no sucesso dos negócios, os projetos \textit{web} deixaram de ser tratados de maneira superficial~\cite{pressman:es11}.

Acompanhando as evoluções do mundo, diversos setores aonde não se imaginava uma maneira de como a \textit{web} poderia ser utilizada, foram obrigados a adotar essa tecnologia para poderem se manter no mercado, se equiparando com a concorrência existente. Se antes era preciso concentrar e controlar todos os sistemas de forma interna e não unificada, atualmente já é possível realizar a terceirização de serviços, como por exemplo: o controle de banco de dados, o gerenciamento de e-mails, o controle de inúmeros \textit{hardwares} de maneira remota, entre outros.

Algumas aplicações passaram a desempenhar um papel muito importante nas organizações, como por exemplo, as aplicações de instituições financeiras, que não toleram nenhum tipo de erro em sua utilização. Assim, os problemas encontrados nas aplicações \textit{Desktop}, também passaram a ser visualizados nas aplicações para a \textit{web}. Alguns fatores como a falta de qualificação e experiência dos desenvolvedores, a não utilização de modelos de processo, a não utilização de métricas para estimativas, se somavam para encadear os problemas encontrados. Além disso, o planejamento incoerente, os métodos obsoletos e inadequados, o não cumprimento de custos e prazos, a falta de documentação e o não cumprimento dos requisitos, dificultavam muito o controle da qualidade das aplicações~\cite{peruch-pg07}.               

Neste contexto de atualizações globais, à medida que a complexidade dos \textit{websites} foi aumentando, eles passaram a ser considerados verdadeiras aplicações na \textit{web}, sendo necessário utilizar os fundamentos da Engenharia \textit{Web}, que pode ser definida como a utilização de conceitos, princípios e métodos da Engenharia de Software, de modo que estabeleçam uma maneira de realizar adaptações referentes às características das aplicações \textit{web}~\cite{beder:ew12}.         	







       