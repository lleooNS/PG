% ==============================================================================
% Projeto de Sistema - Nome do Aluno
% Capítulo 2 - Plataforma de Desenvolvimento
% ==============================================================================
\chapter{Plataforma de Desenvolvimento}
\label{sec-plataforma}

\vitor{As tabelas abaixo devem ser adaptadas às tecnologias e ferramentas utilizadas pelo aluno. Foram já indicadas algumas tecnologias bastante utilizadas em disciplinas e projetos em que estou envolvido.}


%=======================================================================================================
%			Tabela de Plataforma de Desenvolvimento e Tecnologias Utilizadas
%=======================================================================================================

Na Tabela~\ref{tabela-plataforma} são listadas as tecnologias utilizadas no desenvolvimento da ferramenta, bem como o propósito de sua utilização.

\begin{table}[h]
	\centering	
	\vspace{0.5cm}
	\footnotesize
	\caption{Plataforma de Desenvolvimento e Tecnologias Utilizadas}	
	\label{tabela-plataforma}
	\begin{tabular}{|p{1.6cm}|c|p{5cm}|p{6.5cm}|}  \hline 
 		
 		\rowcolor[rgb]{0.8,0.8,0.8} Tecnologia & Versão & Descrição & Propósito \\\hline 

		Apache Groovy & 4.0 & Linguagem de programação multifacetada, opcionalmente dinâmica e tipada criada para a plataforma Java. & Escrita do código-fonte das classes que compõem o sistema. \\\hline

		Hibernate & 5.1.5 & Ferramenta de persistência e consulta objeto/relacional de alta performance. & Mapeamento de classes para tabelas de banco de dados. \\\hline 		
		
		Spring Framework & 4.3.9 & Framework Java que possui o objetivo de facilitar o desenvolvimento de aplicações. & Injeção de dependências e inversão de controle. \\\hline
		
		Spring Boot & 1.5.4 & Ferramenta que agiliza o processo de configuração e publicação de aplicações baseadas em Spring. & Configuração e disponibilização de aplicações. \\\hline
		
		Gradle & 3.5 & Sistema de automatização de builds. & Geração de arquivos de build na linguagem Groovy. \\\hline
		
		Spock & 1.1 & Framework de testes e especificações para aplicativos. & Realização de testes. \\\hline 
	
		Apache Tomcat & 9.0.19 & Servidor de Aplicações para Java EE. & Suporte a execução das tecnologias de Servlets, JDBC DataSourcesRealms, JNDI Resources e JSP, cobrindo a parte da especificação J2EE permitindo que o Java funcione no modo web. \\\hline
		
	\end{tabular}
\end{table}






%=======================================================================================================
%			Tabela de Softwares de Apoio ao Desenvolvimento do Projeto
%=======================================================================================================

\newpage
Na Tabela~\ref{tabela-software} vemos os softwares que apoiaram o desenvolvimento de documentos e também do código fonte.

\begin{table}[h]
	\centering	
	\vspace{0.5cm}
	\caption{Softwares de Apoio ao Desenvolvimento do Projeto}	
	\label{tabela-software}
	\begin{tabular}{|p{3cm}|c|p{5cm}|p{6cm}|}  \hline 
	
 		\rowcolor[rgb]{0.8,0.8,0.8} Tecnologia & Versão & Descrição & Propósito \\\hline 
 		 
		FrameWeb Editor & 1.0 & Ferramenta CASE do método FrameWeb. & Criação dos modelos de Entidades, Aplicação, Persistência e Navegação. \\\hline

		TeX Live  & 2018 & Implementação do \LaTeX & Documentação do projeto arquitetural do sistema. \\\hline       
		
		Texmaker & 5.0.2 & Editor de \LaTeX. &  Escrita da documentação do sistema, sendo usado o \textit{template} \textit{abnTeX}.\footnote{\url{http://www.abntex.net.br}.} \\\hline    

		phpMyAdmin & 5.1.0 & Ferramenta de software livre para administrar banco de dados. & Administração de banco de dados. \\\hline 
		
%		Apache Maven & 3.5 & Ferramenta de gerência/construção de projetos de software. & Obtenção e integração das dependências do projeto. \\\hline
	\end{tabular}
\end{table}

