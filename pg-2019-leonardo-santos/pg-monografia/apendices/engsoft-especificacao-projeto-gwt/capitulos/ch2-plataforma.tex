% ==============================================================================
% Projeto de Sistema - Nome do Aluno
% Capítulo 2 - Plataforma de Desenvolvimento
% ==============================================================================
\chapter{Plataforma de Desenvolvimento}
\label{sec-plataforma}

\vitor{As tabelas abaixo devem ser adaptadas às tecnologias e ferramentas utilizadas pelo aluno. Foram já indicadas algumas tecnologias bastante utilizadas em disciplinas e projetos em que estou envolvido.}


%=======================================================================================================
%			Tabela de Plataforma de Desenvolvimento e Tecnologias Utilizadas
%=======================================================================================================

Na Tabela~\ref{tabela-plataforma} são listadas as tecnologias utilizadas no desenvolvimento da ferramenta, bem como o propósito de sua utilização.

\begin{table}[h]
	\centering	
	\vspace{0.5cm}
	\footnotesize
	\caption{Plataforma de Desenvolvimento e Tecnologias Utilizadas}	
	\label{tabela-plataforma}
	\begin{tabular}{|p{1.6cm}|c|p{5cm}|p{6.5cm}|}  \hline 
 		Tecnologia & Versão & Descrição & Propósito \\\hline 
 		
%		Java EE & 7 & Conjunto de especificação de APIs e tecnologias, que são implementadas por programas servidores de aplicação. & Redução da complexidade do desenvolvimento, implantação e gerenciamento de aplicações Web a partir de seus componentes de infra-estrutura prontos para o uso. \\ \hline

		Java & 8 & Linguagem de programação orientada a objetos e independente de plataforma. & Escrita do código-fonte das classes que compõem o sistema. \\\hline
		
%		EJB & 4.0.9 & API para construção de componentes transacionais gerenciados por \textit{container}. & Implementação das regras de negócio em componentes distribuídos, transacionais, seguros e portáveis. \\\hline
		
%		JPA & 2.1 & API para persistência de dados por meio de mapeamento objeto/relacional. & Persistência dos objetos de domínio sem necessidade de escrita dos comandos SQL. \\\hline
		
%		CDI & 1.1 & API para injeção de dependências. & Integração das diferentes camadas da arquitetura. \\\hline
		
%		Facelets & 2.0 &  API para definição de decoradores (\textit{templates}) integrada ao JSF. & Reutilização da estrutura visual comum às paginas, facilitando a manutenção do padrão visual do sistema. \\\hline
		
		GWT SDK & 2.8.2 &  Conjunto de componentes do \textit{framework} GWT. & Inclusão das bibliotecas principais, do compilador e do servidor de desenvolvimento para escrever aplicativos na web. \\\hline
		
		MySQL Server & 8.0 & Sistema Gerenciador de Banco de Dados Relacional gratuito. & Armazenamento dos dados manipulados pela ferramenta. \\\hline
		
		Apache Tomcat & 9.0.19 & Servidor de Aplicações para Java EE. & Suporte a execução das tecnologias de Servlets, JDBC DataSourcesRealms, JNDI Resources e JSP, cobrindo a parte da especificação J2EE permitindo que o Java funcione no modo web. \\\hline
		
		Java Platform (JDK) & 12 & Conjunto de utilitários para a plataforma Java. & Compilação e inclusão de bibliotecas (API’s) necessárias para criação de programas em Java e ferramentas úteis para o desenvolvimento. \\\hline
		
		Apache Ant & 1.9.14 & Biblioteca Java e ferramenta de comandos que permitem compilar, montar, testar e executar aplicativos Java. & Execução de argumentos de linha de comando. \\\hline
	\end{tabular}
\end{table}






%=======================================================================================================
%			Tabela de Softwares de Apoio ao Desenvolvimento do Projeto
%=======================================================================================================

\newpage
Na Tabela~\ref{tabela-software} vemos os softwares que apoiaram o desenvolvimento de documentos e também do código fonte.

\begin{table}[h]
	\centering	
	\vspace{0.5cm}
	\caption{Softwares de Apoio ao Desenvolvimento do Projeto}	
	\label{tabela-software}
	\begin{tabular}{|p{3cm}|c|p{5cm}|p{6cm}|}  \hline 
	
 		Tecnologia & Versão & Descrição & Propósito \\\hline 
 		 
		FrameWeb Editor & 1.0 & Ferramenta CASE do método FrameWeb. & Criação dos modelos de Entidades, Aplicação, Persistência e Navegação. \\\hline

		TeX Live  & 2018 & Implementação do \LaTeX & Documentação do projeto arquitetural do sistema. \\\hline       
		
		Texmaker & 5.0.2 & Editor de LaTeX. &  Escrita da documentação do sistema, sendo usado o \textit{template} \textit{abnTeX}.\footnote{\url{http://www.abntex.net.br}.} \\\hline    

		Eclipse Java EE IDE for Web Developers & 4.8 & Ambiente de desenvolvimento (IDE) com suporte ao desenvolvimento Java EE. & Implementação, implantação e testes da aplicação Web Java EE. \\\hline 
		
		Apache Maven & 3.5 & Ferramenta de gerência/construção de projetos de software. & Obtenção e integração das dependências do projeto. \\\hline
	\end{tabular}
\end{table}

