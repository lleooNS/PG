\chapter{Atributos de Qualidade e Táticas}
\label{sec-atributos}

Na Tabela~\ref{tabela-atributos} são listados os atributos de qualidade considerados neste projeto, com uma indicação se os mesmos são condutores da arquitetura ou não e as táticas a serem utilizadas para tratá-los.

\begin{table}[h]
	\centering	
	\vspace{0.5cm}
	\caption{Atributos de Qualidade e Táticas Utilizadas}	
	\label{tabela-atributos}
	\begin{tabular}{|p{3.5cm}|p{2cm}|p{1.9cm}|p{7cm}|}  \hline 
	
 		\rowcolor[rgb]{0.8,0.8,0.8} Categoria & Requisitos Não Funcionais & Condutor da Arquitetura & Tática \\\hline
 		
 		Facilidade de Aprendizado, Operabilidade & RNF02, RNF03 & Sim & • Facilitar a compreensão do usuário para que os conceitos chaves sejam entendidos  de forma intuitiva, possibilitando que os comandos do sistema sejam operados com mais eficiência. Para isso, o sistema deve possuir diálogos simples para que nenhuma dúvida seja gerada no momento da inserção das informações. \\\hline  
 		
		Segurança de Acesso & RNF01 & Sim & • Garantir que os dados estão acessíveis apenas para aqueles que estão autorizados a acessá-los, evitando acesso não autorizado ou modificação dos dados.\\ &&&• Autorizar usuários, usando as classes de usuários definidas no projeto. \\\hline
		
		Manutenibilidade, Portabilidade & RNF04, RNF05, RNF06, RNF07 & Sim & • Organizar a arquitetura da ferramenta segundo uma combinação de camadas e partições.\\ &&&• A camada de lógica de negócio deve ser organizada segundo o padrão Camada de Serviço.\\ &&&• A camada de gerência de dados deve ser organizada segundo o padrão DAO.\\ &&&• Separar a interface com o usuário do restante da aplicação, segundo o padrão MVC. \\\hline  
		
		Reusabilidade & RNF07 & Não & • Reutilizar componentes e frameworks existentes.\\ &&&• Desenvolver novos componentes para reuso, quando não houver componentes disponíveis e houver potencial para reuso. \\\hline	
 		
	\end{tabular}
\end{table}
